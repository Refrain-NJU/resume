% !TEX TS-program = xelatex
% !TEX encoding = UTF-8 Unicode
% !Mode:: "TeX:UTF-8"

\documentclass{resume}
\usepackage{zh_CN-Adobefonts_external} % Simplified Chinese Support using external fonts (./fonts/zh_CN-Adobe/)
%\usepackage{zh_CN-Adobefonts_internal} % Simplified Chinese Support using system fonts
\usepackage{linespacing_fix} % disable extra space before next section
\usepackage{cite}

\begin{document}
\pagenumbering{gobble} % suppress displaying page number

\name{谭子熠}

\basicInfo{
  \email{ajb459684460@gmail.com} \textperiodcentered\
  \phone{13840304617} \textperiodcentered\
  \github[Ziy1-Tan]{https://github.com/Ziy1-Tan}}

\section{\faGraduationCap\  教育背景}
\datedsubsection{\textbf{华中科技大学}, 武汉, 湖北}{2021 -- 2024}
\textit{硕士}\ 计算机系统结构
\datedsubsection{\textbf{东北大学}, 沈阳, 辽宁}{2017 -- 2021}
\textit{学士}\ 软件工程


\section{\faUsers\ 实习/项目经历}
\datedsubsection{\underline{\textbf{\href{https://summerofcode.withgoogle.com/programs/2022/projects/VIJfR79a}{MariaDB - Google Summer of Code}}}}{2022.5 -- 2022.9}
% \role{实习(远程)}{Contributor}
Columnstore基于MairaDB构建的大规模并行,高性能,压缩,分布式开源列式存储引擎。本人主要负责Columnstore SELECT子句下JSON Functions的设计与实现。

\begin{onehalfspacing}
  \begin{itemize}
    \item 设计并实现了一系列SELECT子句下JSON Functions,针对不同编码数据均表现良好;
    \item 利用Exception、RAII进行再封装,封装后的资源有具备更强的内聚性和Resource-safe;
    \item 针对JSON和MariaDB Boolean不兼容问题进行适配,使之具有完全兼容性;
    \item 针对JSON Function执行过程中的常量参数进行缓存,提高了函数的执行效率;
    \item 利用Mysql-test-run测试框架对函数进行SQL Test,测试覆盖了函数关键分支和边缘分支;
    \item \textbf{成果:}\github[PR]{https://github.com/mariadb-corporation/mariadb-columnstore-engine/pull/2425}(共8600+行,测试2500+行)成功合入主分支,最终在Columnstore 22.08.2发布;
    \item 开发阶段额外提交的3个patch均被合并:
          \begin{itemize}
            \item \github[MDEV-28947 JSON\_TYPE result is turncated, charset max length should be considered]{https://github.com/MariaDB/server/pull/2172};
            \item \github[MDEV-29264: JSON function overflow error based on LONGTEXT field]{https://github.com/MariaDB/server/pull/2226};
            \item \github[Refactor: remove redundant assignments of JSON\_MERGE\_PATCH]{https://github.com/MariaDB/server/pull/2209};
          \end{itemize}
    \item \textbf{收获:}对于JSON和SQL函数的解析执行过程有了更深的理解,对于UTF-8数据处理有了实践上的理解,对于利用测试保证重构有了初步的认识;
  \end{itemize}
\end{onehalfspacing}

\datedsubsection{\underline{\textbf{\href{https://summer-ospp.ac.cn\#/org/orgdetail/1bf2ea9b-fdf1-45a5-bbb1-d81827445986/}{JuiceFS - 中科院开源软件供应链点亮计划}}}}{2022.6 -- 2022.10}
% \role{实习(远程)}{Contributor}
JuiceFS 是一款「数据」与「元数据」分离的架构高性能POSIX文件系统。本人主要负责doctor子命令的设计与实现。

\begin{onehalfspacing}
  \begin{itemize}
    \item doctor命令可以收集并展示系统各个维度信息,以帮助开发者更好地了解系统并定位错误;
    \item 命令支持收集OS、架构、JuiceFS版本、日志、挂载点配置信息和pprof统计信息等指标;
    \item 利用并发和线程同步技术,将获取prometheus和pprof指标过程异步化,提高命令整体执行效率;
    \item \textbf{成果:}测试覆盖了doctor的关键分支,\github[PR]{https://github.com/juicedata/juicefs/pull/2360}成功合入主分支;
    \item \textbf{收获:}对线程同步、线程通信等多线程技术有了更深的理解,初步了解测试驱动开发(TDD)模式;
  \end{itemize}
\end{onehalfspacing}

\datedsubsection{\underline{\textbf{\href{https://github.com/pingcap/tiflash}{TiFlash Community - PingCAP}}}}{2022.10 -- 至今}
本人主要作为贡献者贡献于TiDB的计算引擎TiFlash
\begin{itemize}
  \item 引入支持enum类型静态反射库magic\_enum,在保证系统整体性能的情况下,缓解enum相关样板代码冗余的情况。\github[PR]{https://github.com/pingcap/tiflash/pull/5843};
  \item 设计并实现OctInt/OctString函数下推(正在进行)\github[Issue]{https://github.com/pingcap/tiflash/issues/5110}
\end{itemize}

\section{\faHeartO\ 获奖情况}
\datedline{\textit{复赛32强(32/478)}, 华为软件精英挑战赛武长赛区}{2022 年4 月}
\datedline{华中科技大学社会实践优秀个人}{2022 年3 月}
\datedline{华中科技大学研究生一等奖学金}{2021 年9 月}

\end{document}

