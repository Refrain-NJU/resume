% !TEX TS-program = xelatex
% !TEX encoding = UTF-8 Unicode
% !Mode:: "TeX:UTF-8"

\documentclass{resume}
\usepackage{zh_CN-Adobefonts_external} % Simplified Chinese Support using external fonts (./fonts/zh_CN-Adobe/)
%\usepackage{zh_CN-Adobefonts_internal} % Simplified Chinese Support using system fonts
\usepackage{linespacing_fix} % disable extra space before next section
\usepackage{cite}

\begin{document}
\pagenumbering{gobble} % suppress displaying page number

\name{谭子熠}

\basicInfo{
  \phone{13840304617} \textperiodcentered\
  \email{ajb459684460@gmail.com} \textperiodcentered\
  \github[jihy]{https://github.com/Ziy1-Tan}
}

\section{\faGraduationCap\  教育背景}
\datedsubsection{\textbf{华中科技大学}, 武汉, 湖北}{2021.9 -- 2024.6}
\textit{硕士}\ 计算机系统结构
\datedsubsection{\textbf{东北大学}, 沈阳, 辽宁}{2017.9 -- 2021.6}
\textit{学士}\ 软件工程, \textit{专业排名9\%, 优秀毕业生}

\section{\faUsers\ 实习/项目经历}
\datedsubsection{\underline{\textbf{\href{https://summerofcode.withgoogle.com/programs/2022/projects/VIJfR79a}{MariaDB - Google Summer of Code}}}\enspace(C++、SQL、JSON)}{2022.5 -- 2022.9}
共有5,155个注册人员, 其中1209份申请被Google接纳\textbf{(中选率23\%)}. ColumnStore是基于MariaDB构建的分布式列式存储引擎.
% 它为MariaDB提供了快速的大规模数据分析能力.
\begin{itemize}[parsep=0.3ex]
  \item \textbf{个人贡献:}设计实现了SELECT子句中的JSON函数, 补全缺失的JSON函数部分;
  \item \textbf{资源封装:}利用Exception和RAII进行封装, 使之具备更强的\textbf{内聚性}和\textbf{资源安全性};
  \item \textbf{类型适配:}适配JSON/MariaDB布尔类型兼容性问题, 确保对JSON类型的全兼容;
  \item \textbf{完备测试:}利用MySQL Test Run框架进行单元测试, 覆盖函数的\textbf{关键和边缘分支};
  \item \textbf{最终成果:} PR(测试2500行/共8600行)成功\textbf{合入主分支}, 最终在Release 22.08.2发布;\enspace\githubicon{https://github.com/mariadb-corporation/mariadb-columnstore-engine/pull/2425}
  \item 提交合并了3个JSON相关patch, \textbf{最大影响版本}10.3\textasciitilde 10.9:
        \begin{itemize}
          \item 修复了由于未考虑编码长度导致JSON\_TYPE结果被截断的问题;\enspace\githubicon{https://github.com/MariaDB/server/pull/2172}
          \item 修复了由于uint\_32类型溢出导致JSON\_*结果被截断的问题;\enspace\githubicon{https://github.com/MariaDB/server/pull/2226}
          \item 重构了JSON\_MERGE\_PATCH, 移除其中的冗余赋值操作;\enspace\githubicon{https://github.com/MariaDB/server/pull/2209}
        \end{itemize}
  \item \textbf{收获:}扩宽了复杂问题的排查思路, 对处理多编码数据处理的实践理解, 以及使用测试确保重构的初步认识.
\end{itemize}
\datedsubsection{\underline{\textbf{\href{https://summer-ospp.ac.cn\#/org/orgdetail/1bf2ea9b-fdf1-45a5-bbb1-d81827445986/}{JuiceFS - 中科院开源之夏}}}\enspace(Go、多线程、文件系统)}{2022.6 -- 2022.10}
JuiceFS是数据/元数据分离的分布式文件系统. JuiceFS高度用户友好且易于部署, 对象存储服务和元数据引擎可根据用例灵活配置切换.
\begin{itemize}[parsep=0.3ex]
  \item \textbf{个人贡献:}设计实现了doctor诊断子命令, 以帮助更好地定位错误;
  \item \textbf{多元采集:}支持内核版本/FS/挂载配置等信息的收集, 以及日志/pprof等性能指标的采样;
  \item \textbf{流程优化:}利用并发技术, 完成指标采样异步化改造, \textbf{保证执行效率和用户体验};
  \item \textbf{最终成果:}测试覆盖了doctor的关键分支, PR成功\textbf{合入主分支};\enspace\githubicon{https://github.com/juicedata/juicefs/pull/2360}
  \item \textbf{收获:}加深了对线程同步等并发技术的理解, 初步了解测试驱动开发(TDD)模式.
\end{itemize}
\datedsubsection{\underline{\textbf{\href{https://github.com/opencurve/curve}{Curve - CNCF Sandbox Project}}}}{2022.12 -- 至今}
网易数帆团队开源的分布式存储系统, 支持文件和块存储。与JuiceFS不同, Curve内置元数据引擎.
\begin{itemize}[parsep=0.3ex]
  \item \textbf{回滚支持:}为创建FS操作添加回滚支持, 保证了资源的及时释放, \textbf{增量分支覆盖率$\geq$80\%};\enspace\githubicon{https://github.com/opencurve/curve/pull/2206}
  \item \textbf{流程优化:}剔除冗余上传, 利用并发等待cache上传完毕, \textbf{理想情况下}节省50\%空间和调用.\enspace\githubicon{https://github.com/opencurve/curve/pull/2252}
\end{itemize}
\section{\faCogs\ 相关技能}
\begin{itemize}[parsep=0.5ex]
  \item \textbf{编程语言:}熟悉C++($\geq${5000})、Go/Python(1000\textasciitilde5000)、Shell($\leq$1000)
  \item \textbf{框架:}有使用gtest进行单元和插桩测试的经验, 有brpc、git、gdb、vim的使用经验
\end{itemize}
\section{\faHeartO\ 获奖情况}
\datedline{华中科技大学腾讯奖学金}{2022 年12 月}
\datedline{\textit{二等奖(获奖率6.6\%)}, 华为软件精英挑战赛武长赛区}{2022 年4 月}
\datedline{华中科技大学社会实践优秀个人}{2022 年3 月}
\datedline{华中科技大学研究生一等奖学金}{2021 年9 月}
\end{document}

