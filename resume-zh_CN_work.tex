% !TEX TS-program = xelatex
% !TEX encoding = UTF-8 Unicode
% !Mode:: "TeX:UTF-8"

\documentclass{resume}
\usepackage{zh_CN-Adobefonts_external} % Simplified Chinese Support using external fonts (./fonts/zh_CN-Adobe/)
% \usepackage{zh_CN-Adobefonts_internal} % Simplified Chinese Support using system fonts
\usepackage{linespacing_fix} % disable extra space before next section
\usepackage{cite}

\begin{document}
\pagenumbering{gobble} % suppress displaying page number

\name{谭子熠}
\basicInfo{
男 \textperiodcentered\
1998-09 \textperiodcentered\
中共预备党员 \textperiodcentered\
广西百色
}

\basicInfo{
  \phone{(+86)138-4030-4617} \textperiodcentered\
  \email{tanziyi0925@gmail.com} \textperiodcentered\
  \github[Ziy1-Tan]{https://github.com/Ziy1-Tan}
}

\section{\faGraduationCap\ 教育背景}
\datedsubsection{\textbf{华中科技大学(保研)} \enspace 武汉,湖北}{2021.9 -- 2024.6}
\textit{工学硕士}\ 计算机系统结构 \enspace \textit{校一等奖学金、腾讯奖学金、华为软挑二等奖(获奖率6.6\%)}
\datedsubsection{\textbf{东北大学} \enspace 沈阳,辽宁}{2017.9 -- 2021.6}
\textit{工学学士}\ 软件工程 \enspace \textit{专业排名9\%, 刘大阶方文玉奖学金、优秀毕业生}

\section{\faBuilding\ 实习经历}

\datedsubsection{\textbf{\href{https://www.aliyun.com/product/ons}{消息中间件团队 - 阿里巴巴云智能集团}}\enspace(研发实习生)}{2023.7 -- 2023.9}
负责消息产品MNS和RocketMQ限流接入、性能回归和社区贡献,致力于客户提供高可用的消息服务.
\begin{itemize}[parsep=0.2ex]
  \item 实现了基于令牌桶算法的单机-集群\textbf{多级限流方案}, 有效保证MNS单实例/集群TPS稳定性;
  \item 采用了一致性哈希的限流上报策略与限流计数节点集群部署模式, \textbf{保证集群限流服务的高可用性};
  \item 设计了多级限流开关和白名单机制, 确保限流功能\textbf{平滑迁移上线}, 保证存量用户体验;
  \item 梳理评估了潜在限流降级风险点, 针对性地对进行了容灾/时延/长稳测试, \textbf{效果均达预期};
  \item MNS集群限流于9.5日成功\textbf{上线广州Region}, 运行状况良好, 计划陆续上线其他Region;
  \item 负责RocketMQ内核升级的性能回归和兼容适配, 针对海量Topic场景提出\textbf{元数据检索, 批量路由注册}等兼容性改进, 为多产品内核统一提供前置支持. \enspace\githubicon{https://github.com/apache/rocketmq/pull/7276}\githubicon{https://github.com/apache/rocketmq/pull/7325}
\end{itemize}

\section{\faUsers\ 开源经历}

\datedsubsection{\textbf{\href{https://summerofcode.withgoogle.com/programs/2022/projects/VIJfR79a}{MariaDB - Google开源之夏}}\enspace (C++、JSON、列式存储)}{2022.6 -- 2022.9}
ColumnStore是基于MariaDB构建的分布式列式存储引擎. 已连续举办18年全球知名开源活动, 2022年共5,155次注册, 1209份申请被Google接纳, 大陆仅40人左右中选\textbf{(中选率3.3\%)}.
\begin{itemize}[parsep=0.2ex]
  \item \textbf{个人贡献:}设计并实现了\verb|SELECT|子句中的JSON函数, 完成了MariaDB的JSON函数下推;
  \item \textbf{资源封装:}利用RAII进行资源封装, 使之具备更强的\textbf{内聚性}和\textbf{资源安全性};
  \item \textbf{变量优化:}完成\verb|malloc|到\verb|alloca|分配的部分迁移, 保证\textbf{小字符串处理}的效率和内存安全性;
  \item \textbf{类型适配:}适配JSON/MariaDB布尔类型兼容性问题, 系统间JSON类型传递完全打通;
  \item \textbf{完备测试:}利用内置的测试框架MTR进行全面的单元测试, 测试覆盖了函数的\textbf{关键和边缘分支};
  \item \textbf{最终成果:} 8000+代码成功\textbf{合并到代码主分支}, 最终在Release 22.08.2发布;\enspace\githubicon{https://github.com/mariadb-corporation/mariadb-columnstore-engine/pull/2425}
  \item 提交合并了3个JSON相关补丁, 确保用户在10.3\textasciitilde 10.9版本的兼容性和稳定性:\enspace\githubicon{https://github.com/MariaDB/server/pull/2172}\githubicon{https://github.com/MariaDB/server/pull/2226}\githubicon{https://github.com/MariaDB/server/pull/2209}
\end{itemize}

\datedsubsection{\textbf{\href{https://summer-ospp.ac.cn\#/org/orgdetail/1bf2ea9b-fdf1-45a5-bbb1-d81827445986/}{JuiceFS - 中科院开源之夏}}\enspace(Golang、文件系统、并发)}{2022.7 -- 2022.10}
JuiceFS是数据/元数据分离的分布式文件系统, 用户友好且易于部署, 存储和元数据服务可灵活配置切换.
\begin{itemize}[parsep=0.2ex]
  \item \textbf{个人贡献:}设计实现了doctor诊断子命令, 执行\verb|juicefs doctor ...|帮助更好地定位错误;
  \item \textbf{多元采集:}支持Kernel、FS、Mount等信息的收集, 以及log、pprof等运行时指标的采样;
  \item \textbf{流程优化:}利用并发技术, 完成指标采样异步化改造, \textbf{保证执行效率和用户体验};
  \item \textbf{最终成果:}测试覆盖了doctor的关键分支, PR成功\textbf{合入主分支};\enspace\githubicon{https://github.com/juicedata/juicefs/pull/2360}
\end{itemize}

\section{\faCogs\ 相关技能}
\begin{itemize}
  \item 擅长使用C++语言和新标准, 熟悉STL、OOP使用和原理, 了解Golang、Python;
  \item 擅长使用Git开发工作流, 熟悉GDB使用, 能够利用断点、变量等命令进行问题定位;
  \item 熟悉GoogleTest, Mock测试套件, 遵循测试保证重构思想, 了解测试驱动开发(TDD)流程;
  \item 熟悉Linux使用, 能使用常见监控命令了解服务状态, 了解Shell, 有基本的脚本思维;
\end{itemize}

\section{\faHeartO\ 荣誉情况}
\datedline{中国开源创新大赛二等奖}{2023年5月}
\datedline{华中科技大学社会活动积极分子}{2022年11月}
\datedline{华中科技大学校二等奖学金}{2022年11月}
\datedline{华中科技大学社会实践优秀个人}{2022年3月}
\end{document}

