% !TEX TS-program = xelatex
% !TEX encoding = UTF-8 Unicode
% !Mode:: "TeX:UTF-8"

\documentclass{resume}
\usepackage{zh_CN-Adobefonts_external} % Simplified Chinese Support using external fonts (./fonts/zh_CN-Adobe/)
%\usepackage{zh_CN-Adobefonts_internal} % Simplified Chinese Support using system fonts
\usepackage{linespacing_fix} % disable extra space before next section
\usepackage{cite}

\begin{document}
\pagenumbering{gobble} % suppress displaying page number

\name{谭子熠}

\basicInfo{
  \email{ajb459684460@gmail.com} \textperiodcentered\
  \phone{13840304617} \textperiodcentered\
  \github[Ziy1-Tan]{https://github.com/Ziy1-Tan}}

\section{\faGraduationCap\  教育背景}
\datedsubsection{\textbf{华中科技大学}, 武汉, 湖北}{2021 -- 2024}
\textit{硕士}\ 计算机系统结构
\datedsubsection{\textbf{东北大学}, 沈阳, 辽宁}{2017 -- 2021}
\textit{学士}\ 软件工程

% Reference Test
%\datedsubsection{\textbf{Paper Title\cite{zaharia2012resilient}}}{May. 2015}
%An xxx optimized for xxx\cite{verma2015large}
%\begin{itemize}
%  \item main contribution
%\end{itemize}

% \section{\faCogs\ 技能}
% % increase linespacing [parsep=0.5ex]
% \begin{itemize}[parsep=0.5ex]
%   \item 熟悉C/C++编程语言,了解常见STL容器的使用和实现,了解Go、Python等;
%   \item 熟悉Linux环境下常用命令和相关工具的使用(Gdb、Git、Vim等);
%   \item 具有Linux环境下大型C++系统编译、调试和开发经验,能够快速定位并开发;
% \end{itemize}

\section{\faUsers\ 实习/项目经历}
\datedsubsection{\textbf{\href{https://summerofcode.withgoogle.com/programs/2022/projects/VIJfR79a}{MariaDB - Google Summer of Code}}}{2022.5 -- 2022.9}
% \role{实习(远程)}{Contributor}
Columnstore(MCS) 是一个大规模的并行分布式数据架构的列式存储引擎。个人主要负责MCS Select Handler下分布式JSON Functions的设计与实现。
\begin{itemize}
  \item 设计并实现了31个MCS SELECT子句下JSON Functions,与MariaDB中表现一致;
  \item 针对JSON和MariaDB Boolean类型不兼容问题进行适配,使之具有完全兼容性;
  \item 利用Exception、RAII等特性进行封装,封装后的代码有更强的exception-safe和resource-safe;
  \item 基于缓存原理对函数执行过程的常量值进行缓存,大大提高了函数的执行效率;
  \item 利用Mysql-test-run测试框架对函数进行SQL Test,测试覆盖了函数的关键分支;
  \item \textbf{成果:}特性\github[MCOL-785 Implement DISTRIBUTED JSON functions]{https://github.com/mariadb-corporation/mariadb-columnstore-engine/pull/2425}成功合入主分支并发布;
  \item 额外patch均被合并:
        \begin{itemize}
          \item \github[Refactor: remove redundant assignments of JSON\_MERGE\_PATCH]{https://github.com/MariaDB/server/pull/2209};
          \item \github[MDEV-28947 JSON\_TYPE result is turncated, charset max length should be considered]{https://github.com/MariaDB/server/pull/2172};
          \item \github[MDEV-29264: JSON function overflow error based on LONGTEXT field]{https://github.com/MariaDB/server/pull/2226};
        \end{itemize}
  \item \textbf{收获:}对于SELECT查询计划的生成有了一定的了解,对于C/C++代码的编译和调试更加熟练,对于测试用例设计有了进一步了解;
\end{itemize}

\datedsubsection{\textbf{\href{https://summer-ospp.ac.cn/}{JuiceFS - 中科院开源软件供应链点亮计划}}}{2022.6 -- 2022.10}
% \role{实习(远程)}{Contributor}
JuiceFS 是一款「数据」与「元数据」分离存储的架构高性能POSIX文件系统。个人主要负责JuiceFS doctor子命令的设计与实现。

\begin{onehalfspacing}
\begin{itemize}
    \item 设计并实现了doctor子命令,收集系统各个维度信息,以更好地定位错误;
    \item 支持收集OS、架构、JuiceFS版本、日志、挂载点配置信息和pprof统计信息等指标;
    \item 利用并发和线程同步技术,将获取prometheus、pprof过程异步化,提高了doctor整体执行效率;
    \item \textbf{成果:}测试覆盖了doctor的关键分支,PR \github[cmd/doctor: doctor command...]{https://github.com/juicedata/juicefs/pull/2360}即将合入主分支;
    \item \textbf{收获:}对线程同步、线程通信和线程池等多线程并发技术有了更深的理解,对文件存储有了进一步的理解,初步了解测试驱动开发(TDD)模式;
\end{itemize}
\end{onehalfspacing}

\datedsubsection{\textbf{TiFlash Community}}{2022.9 -- 至今}
% \role{TiFlash}{Contributor}

TiFlash Contributor
\begin{itemize}
    \item 贡献于TiDB的列式分析引擎TiFlash。引入支持enum类型静态反射库magic\_enum,简化现有enum转换代码而不损失性能。\github[PR]{https://github.com/pingcap/tiflash/pull/5843};
    \item 设计并实现OctInt/OctString函数下推(目前正在进行)\github[Issue]{https://github.com/pingcap/tiflash/issues/5110}
\end{itemize}

\section{\faHeartO\ 获奖情况}
\datedline{\textit{复赛32强(32/478)}, 华为软件精英挑战赛武长赛区}{2022 年4 月}
\datedline{华中科技大学社会实践优秀个人}{2022 年3 月}
\datedline{华中科技大学研究生一等奖学金}{2021 年9 月}


% \section{\faInfo\ 其他}
% % increase linespacing [parsep=0.5ex]
% \begin{itemize}[parsep=0.5ex]
%   \item 语言: CET6
% \end{itemize}

% %% Reference
% %\newpage
% %\bibliographystyle{IEEETran}
% %\bibliography{mycite}
\end{document}

