% !TEX TS-program = xelatex
% !TEX encoding = UTF-8 Unicode
% !Mode:: "TeX:UTF-8"

\documentclass{resume}
\usepackage{zh_CN-Adobefonts_external} % Simplified Chinese Support using external fonts (./fonts/zh_CN-Adobe/)
%\usepackage{zh_CN-Adobefonts_internal} % Simplified Chinese Support using system fonts
\usepackage{linespacing_fix} % disable extra space before next section
\usepackage{cite}

\begin{document}
\pagenumbering{gobble} % suppress displaying page number

\name{谭子熠}

\basicInfo{
  \email{ajb459684460@gmail.com} \textperiodcentered\ 
  \phone{(+86) 13840304617} \textperiodcentered\ 
  \github[Ziy1-Tan]{https://github.com/Ziy1-Tan}}
 
\section{\faGraduationCap\  教育背景}
\datedsubsection{\textbf{华中科技大学}, 武汉, 湖北}{2021 -- 2024}
\textit{硕士}\ 计算机系统结构
\datedsubsection{\textbf{东北大学}, 沈阳, 辽宁}{2017 -- 2021}
\textit{学士}\ 软件工程

% Reference Test
%\datedsubsection{\textbf{Paper Title\cite{zaharia2012resilient}}}{May. 2015}
%An xxx optimized for xxx\cite{verma2015large}
%\begin{itemize}
%  \item main contribution
%\end{itemize}

\section{\faCogs\ 技能}
% increase linespacing [parsep=0.5ex]
\begin{itemize}[parsep=0.5ex]
  \item 熟悉C/C++编程语言,了解Go、Python等;
  \item 熟悉Linux环境下常用命令和相关工具的使用(gdb、git、vim等);
  \item 熟悉论文阅读,能够快速梳理文章的gap、related work和insight;
  \item 了解latex、dblp、google scholar等科研工具使用;
  \item 具有Linux环境下大型系统编译、调试和开发经验,具有存储系统开发经验;
\end{itemize}

\section{\faUsers\ 实习/项目经历}
\datedsubsection{\textbf{\href{https://summerofcode.withgoogle.com/programs/2022/projects/VIJfR79a}{MariaDB - Google Summer of Code}} 武汉}{2022.5 -- 2022.9}
% \role{实习(远程)}{Contributor}
Columnstore(MCS) 是一个大规模的并行分布式数据架构的列式存储引擎。个人主要负责MCS Select Handler下分布式JSON Functions的设计与实现。
\begin{itemize}
  \item \textbf{效果:}支持在MCS的Select SQL语句中使用JSON Functions;
  \item 针对JSON和MariaDB Boolean不完全兼容问题进行处理,改造后具有完全兼容性;
  \item 利用Exception、RAII等技术对代码进行封装,封装后具备更强的exception-safe和resource-safe
  \item 基于缓存技术缓存函数执行过程中的常量值,大大提高了函数执行效率;
  \item 利用Mysql-test-run测试框架对函数进行SQL Test,最终测试覆盖率达到90\%
  \item \textbf{成果:}\github[PR]{https://github.com/mariadb-corporation/mariadb-columnstore-engine/pull/2425}(8600行,其中测试2000行)成功合入主分支
  \item 开发过程额外提交的3个patch均被合并:
  \begin{itemize}
    \item \github[Refactor: remove redundant assignments of JSON\_MERGE\_PATCH]{https://github.com/MariaDB/server/pull/2209}
    \item \github[MDEV-28947 JSON\_TYPE result is turncated, charset max length should be considered]{https://github.com/MariaDB/server/pull/2172}
    \item \github[MDEV-29264: JSON function overflow error based on LONGTEXT field]{https://github.com/MariaDB/server/pull/2226}
  \end{itemize}
  \item \textbf{收获:}对于JSON和SQL函数解析执行过程有了更深的理解,对于C/C++代码的编译和调试更加熟练,对于测试用例编写有了一定的了解

\end{itemize}

\datedsubsection{\textbf{\href{https://summer-ospp.ac.cn/}{JuiceFS - 中科院开源软件供应链点亮计划}} 武汉}{2022.6 -- 2022.10}
% \role{实习(远程)}{Contributor}
JuiceFS 是一款「数据」与「元数据」分离存储的架构高性能POSIX文件系统。个人主要负责JuiceFS doctor子命令的设计与实现。
\begin{onehalfspacing}
\begin{itemize}
  \item \textbf{效果:}用户可以通过doctor子命令收集系统各个维度信息,以更好地定位错误
  \item 支持收集OS、架构、JuiceFS版本、日志、元数据引擎、挂载点配置和pprof统计信息等
  \item \textbf{成果:}测试覆盖率90\%,项目通过最终评估,\github[PR]{https://github.com/juicedata/juicefs/pull/2360}成功合入主分支
  \item \textbf{收获:}对分布式文件存储架构有了进一步的理解,对线程同步、线程通信和线程池等多线程并发技术有了进一步的认识,初步了解测试驱动开发(TDD)模式。
\end{itemize}
\end{onehalfspacing}


\section{\faHeartO\ 获奖情况}
\datedline{\textit{复赛32强(32/478)}, 华为软件精英挑战赛武长赛区}{2022 年4 月}
\datedline{华中科技大学社会实践优秀个人}{2022 年3 月}
\datedline{华中科技大学研究生一等奖学金}{2021 年9 月}


% \section{\faInfo\ 其他}
% % increase linespacing [parsep=0.5ex]
% \begin{itemize}[parsep=0.5ex]
%   \item 语言: CET6
% \end{itemize}

% %% Reference
% %\newpage
% %\bibliographystyle{IEEETran}
% %\bibliography{mycite}
\end{document}

